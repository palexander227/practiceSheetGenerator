\begin{tikzpicture}
\draw[gray!60!black,<->] (0.2,2.64)parabola bend (-1.5,-0.25)(-3.2,2.64);
\draw[<->](-2.5,0)--(2.5,0);
\draw[<->](0,-2)--(0,3);
\end{tikzpicture}

The graph above is $f(x) = a(x + 1)(x + 2)$, where $a$
is a constant.  If $g(x) = -a(x - 1)(x - 2)$ is
graphed on the same axes, which of the following
describes the transformation of $f(x)$ to $g(x)$?
\begin{itemize}
\item I   The graph moves right 3 units.
\item II  The graph moves left 3 units.
\item III The graph is reflected across the x axis.
\end{itemize}


\ifsat
	\begin{enumerate}[label=\Alph*)]
		\item I only
		\item III only
		\item I and III only%
		\item II and III only
	\end{enumerate}
\else
\fi

\ifacteven
	\begin{enumerate}[label=\textbf{\Alph*.},itemsep=\fill,align=left]
		\setcounter{enumii}{5}
		\item I only
		\item II only
		\item III only
		\addtocounter{enumii}{1}
		\item I and III only%
		\item II and III only
	\end{enumerate}
\else
\fi

\ifactodd
	\begin{enumerate}[label=\textbf{\Alph*.},itemsep=\fill,align=left]
		\item I only
		\item II only
		\item III only
		\item I and III only%
		\item II and III only
	\end{enumerate}
\else
\fi

\ifgridin
 I and III only%
		
\else
\fi

