 
Which of the following represents a way of rewriting $y=2x^2-x-10$ such that the roots of the equation are explicitly stated as values inside the equation?


\ifsat
	\begin{enumerate}[label=\Alph*)]
		\item $y=2(x-\frac{1}{4})^2-\frac{81}{8} $ 
		\item $y=(2x-5)(x+2) $ 
		\item $y=2(x+\frac{5}{2})(x-2) $ 
		\item $y=2(x-\frac{5}{2})(x+2) $ % 
	\end{enumerate}
\else
\fi

\ifacteven
	\begin{enumerate}[label=\textbf{\Alph*.},itemsep=\fill,align=left]
		\setcounter{enumii}{5}
		\item $y=2(x-\frac{1}{4})^2-\frac{81}{8} $ 
		\item $y=(2x-5)(x+2) $ 
		\item $y=2(x+\frac{5}{2})(x-2) $ 
		\addtocounter{enumii}{1}
		\item $y=2(x-\frac{5}{2})(x+2) $ % 
		\item None of these. 
	\end{enumerate}
\else
\fi

\ifactodd
	\begin{enumerate}[label=\textbf{\Alph*.},itemsep=\fill,align=left]
		\item $y=2(x-\frac{1}{4})^2-\frac{81}{8} $ 
		\item $y=(2x-5)(x+2) $ 
		\item $y=2(x+\frac{5}{2})(x-2) $ 
		\item $y=2(x-\frac{5}{2})(x+2) $ % 
		\item None of these. 
	\end{enumerate}
\else
\fi

\ifgridin
 $y=2(x-\frac{5}{2})(x+2) $ % 

\else
\fi

