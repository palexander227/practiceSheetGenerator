 
In the social sciences and in biology, game theory is used to predict the behavior of actors that have to interact with one another.  Ordered pairs are used to communicate how much a particular actor prefers or dislikes given potential outcomes.  The game theory table below is used to reflect the possible outcomes of an evening where Vi and Sheila have three different group activities they could participate in but are asked independently and simultaneously which they would prefer.  In the ordered pairs below, Vi's satisfactions with each particular outcome are the $y$ coordinates and Sheila's satisfactions are the $x$ coordinates.\\
\begin{tabular}{|rr|c|c|c|}\cline{3-5}
\multicolumn{2}{c|}{}&\multicolumn{3}{c|}{Vi\rule{0mm}{0.4cm}}\\
\multicolumn{2}{c|}{}&\rule{0mm}{0.6cm}\pbox{0.5 in}{Go to Club}&\pbox{0.5 in}{Play Frisbee}&\pbox{0.5 in}{See A Play}\\\hline
\multirow{3}{*}{Sheila\rule{0mm}{0.4cm}}&Go to Club\rule{0mm}{0.4cm}& $(8,9)$ & $(6,6)$ & $(-6,-1 )$ \\\cline{3-5}
&Play Frisbee\rule{0mm}{0.4cm} & $(5,7)$&$(9,8)$&$(-6,-1)$\\\cline{3-5}
&See a Play\rule{0mm}{0.4cm}& $(-6,7)$&$(-6,6)$&$(-1,3)$\\\hline
\end{tabular}\\\rule{0mm}{4mm} \\
If Vi reasons that they both won't go to see the play, what would be her average satisfaction with chosing by flipping a fair coin to decide if she will go to the club or play frisbee?\\\\


\ifsat
	\begin{enumerate}[label=\Alph*)]
	\end{enumerate}
\else
\fi

\ifacteven
	\begin{enumerate}[label=\textbf{\Alph*.},itemsep=\fill,align=left]
	\end{enumerate}
\else
\fi

\ifactodd
	\begin{enumerate}[label=\textbf{\Alph*.},itemsep=\fill,align=left]
	\end{enumerate}
\else
\fi

\ifgridin
$7.5$ or $\frac{15}{2}$
\else
\fi

