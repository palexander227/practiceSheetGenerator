 
\begin{center}\begin{tikzpicture}[scale=1.05,rotate=5]
\draw(240:2)--(60:2)node[anchor=south west]{$\beta$};
\draw(-80:2)--(100:2)node[anchor=south]{$\alpha$};
\draw(-2,0)--(2,0)node[anchor=west]{$\gamma$};
\draw(0,0)node[anchor=south east]{$a\degree$} 
node[anchor=north west]{$x\degree$}
node[anchor=north east]{$z\degree$};
\draw(.1,0)node[anchor=south west]{$c\degree$};
\draw(0.12,.3)node[anchor=south]{$b\degree$};
\draw(-0.12,-.3)node[anchor=north]{$y\degree$};
\end{tikzpicture}\\
Note: Figure is not drawn to scale.\end{center}\vspace{-3mm}
In the figure above lines $\alpha, \beta$, and $\gamma$ intersect at a point.  If $b-c=x-z$, which of the following must be true?\\ \\
\begin{enumerate*}[label=\hspace{2mm}\Roman*.]
\item $a=y$ \item $b=x$ \item $c=z$
\end{enumerate*}  


\ifsat
	\begin{enumerate}[label=\Alph*)]
		\item I and II only. 
		\item II and III only.  
		\item I, II, and III. % 
		\item None of the above.
	\end{enumerate}
\else
\fi

\ifacteven
	\begin{enumerate}[label=\textbf{\Alph*.},itemsep=\fill,align=left]
		\setcounter{enumii}{5}
		\item I and II only. 
		\item II and III only.  
		\item I and III only. 
		\addtocounter{enumii}{1}
		\item I, II, and III. % 
		\item None of the above.
	\end{enumerate}
\else
\fi

\ifactodd
	\begin{enumerate}[label=\textbf{\Alph*.},itemsep=\fill,align=left]
		\item I and II only. 
		\item II and III only.  
		\item I and III only. 
		\item I, II, and III. % 
		\item None of the above.
	\end{enumerate}
\else
\fi

\ifgridin
 I, II, and III. % 
		
\else
\fi

