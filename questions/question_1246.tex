 
An economist is running a simulation with fictitious currencies to gain insights into international currency trading.  After $19$ rounds, the relationship between the value of two particular currencies, the krupit and the bilibruble, begins to stabilize. On round $20$, a krupit was worth $2$ bilibrubles.  Each round after that, it took $0.03$ more bilirubles than the previous round to trade for a krupit.  If $t$ represents the number of turns since round $20$, which of the following inequalities describes the set of turns after turn $20$ where it takes more than $4$ bilirubles to trade for $1$ krupit?


\ifsat
	\begin{enumerate}[label=\Alph*)]
		\item $2+0.03t>4 $ % 
		\item $4>2+0.03 $ 
		\item $2>0.03t $ 
		\item $4>2+0.03t $
	\end{enumerate}
\else
\fi

\ifacteven
	\begin{enumerate}[label=\textbf{\Alph*.},itemsep=\fill,align=left]
		\setcounter{enumii}{5}
		\item $2+0.03t>4 $ % 
		\item $4>2+0.03 $ 
		\item $2>0.03t $ 
		\addtocounter{enumii}{1}
		\item $4>2+0.03t $
		\item None of these. 
	\end{enumerate}
\else
\fi

\ifactodd
	\begin{enumerate}[label=\textbf{\Alph*.},itemsep=\fill,align=left]
		\item $2+0.03t>4 $ % 
		\item $4>2+0.03 $ 
		\item $2>0.03t $ 
		\item $4>2+0.03t $
		\item None of these. 
	\end{enumerate}
\else
\fi

\ifgridin
 $2+0.03t>4 $ % 
		
\else
\fi

