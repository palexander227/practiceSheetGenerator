 
$$A=\frac{F}{100}(20-\sqrt[3]{V})$$
An automobile's acceleration when the gas pedal is completely pressed, $A$, is determined by the formula above.  $V$ is the speed of the car, and $F$ is the fuel grade of the gas in the tank.  Which of the following equations represents the fuel grade of the gas in the tank in terms of the other variables? 


\ifsat
	\begin{enumerate}[label=\Alph*)]
		\item {\Large$F=\frac{100A(400+40\sqrt[3]{V}+\sqrt[3]{V^2})}{8000-V} $ } % 
		\item {\Large$F=\frac{100A(400-40\sqrt[3]{V}+\sqrt[3]{V^2})}{8000+V} $ }
		\item {\Large$F=\frac{100A(20+\sqrt[3]{V})}{400-V} $ }
		\item {\Large$F=\frac{100A(400+\sqrt[3]{V})}{8000-V} $}
	\end{enumerate}
\else
\fi

\ifacteven
	\begin{enumerate}[label=\textbf{\Alph*.},itemsep=\fill,align=left]
		\setcounter{enumii}{5}
		\item {\Large$F=\frac{100A(400+40\sqrt[3]{V}+\sqrt[3]{V^2})}{8000-V} $ } % 
		\item {\Large$F=\frac{100A(400-40\sqrt[3]{V}+\sqrt[3]{V^2})}{8000+V} $ }
		\item {\Large$F=\frac{100A(20+\sqrt[3]{V})}{400-V} $ }
		\addtocounter{enumii}{1}
		\item {\Large$F=\frac{100A(400+\sqrt[3]{V})}{8000-V} $}
		\item None of these. 
	\end{enumerate}
\else
\fi

\ifactodd
	\begin{enumerate}[label=\textbf{\Alph*.},itemsep=\fill,align=left]
		\item {\Large$F=\frac{100A(400+40\sqrt[3]{V}+\sqrt[3]{V^2})}{8000-V} $ } % 
		\item {\Large$F=\frac{100A(400-40\sqrt[3]{V}+\sqrt[3]{V^2})}{8000+V} $ }
		\item {\Large$F=\frac{100A(20+\sqrt[3]{V})}{400-V} $ }
		\item {\Large$F=\frac{100A(400+\sqrt[3]{V})}{8000-V} $}
		\item None of these. 
	\end{enumerate}
\else
\fi

\ifgridin
 {\Large$F=\frac{100A(400+40\sqrt[3]{V}+\sqrt[3]{V^2})}{8000-V} $ } % 
		
\else
\fi

