 
The number of full-time employments (FTEs) assigned to art teachers employed by a particular school district from 2005 to 2012 can be modeled by the following equation: $y=-0.35x+8.75$.  In this equation, $x$ represents the number of years since 2005 and $y$ represents the number of full-time employments.  Which of the following best describes the meaning of the number $8.75$ in the equation?


\ifsat
	\begin{enumerate}[label=\Alph*)]
		\item The estimated increase in art teacher FTEs per year
		\item The difference between the number of FTEs in 2012 and the number of FTEs in 2005
		\item The number of full-time employments assigned by the district in 2005 % 
		\item The total number of art teachers employed by the district in 2005
	\end{enumerate}
\else
\fi

\ifacteven
	\begin{enumerate}[label=\textbf{\Alph*.},itemsep=\fill,align=left]
		\setcounter{enumii}{5}
		\item The estimated increase in art teacher FTEs per year
		\item The difference between the number of FTEs in 2012 and the number of FTEs in 2005
		\item The number of full-time employments assigned by the district in 2005 % 
		\addtocounter{enumii}{1}
		\item The total number of art teachers employed by the district in 2005
		\item None of these. 
	\end{enumerate}
\else
\fi

\ifactodd
	\begin{enumerate}[label=\textbf{\Alph*.},itemsep=\fill,align=left]
		\item The estimated increase in art teacher FTEs per year
		\item The difference between the number of FTEs in 2012 and the number of FTEs in 2005
		\item The number of full-time employments assigned by the district in 2005 % 
		\item The total number of art teachers employed by the district in 2005
		\item None of these. 
	\end{enumerate}
\else
\fi

\ifgridin
 The number of full-time employments assigned by the district in 2005 % 
		
\else
\fi

