 
Consider the case where both countries consider themselves as strong as the other and have no weapons capable of truly destroying the other.  While the probability that something will happen that could spark a conflict increases over time, the chance lack of recent provocation will make an event easier to ignore for the sake of peace also increases.  To represent this, a political scientist uses the following equations for $V$ and $T$:
$$V(y)=\frac{y}{y+3} \hspace{2cm} T(y)=\frac{36}{(y+5)^2}$$
In these equations, $y$ represents the time since the last incident in years.  When $E=1$ and the last incident happened $7$ years ago, what is the chance an event between these two countries will lead to war?
\\\\


\ifsat
	\begin{enumerate}[label=\Alph*)]
	\end{enumerate}
\else
\fi

\ifacteven
	\begin{enumerate}[label=\textbf{\Alph*.},itemsep=\fill,align=left]
	\end{enumerate}
\else
\fi

\ifactodd
	\begin{enumerate}[label=\textbf{\Alph*.},itemsep=\fill,align=left]
	\end{enumerate}
\else
\fi

\ifgridin
$\frac{7}{40} $ or $.175 $
\else
\fi

